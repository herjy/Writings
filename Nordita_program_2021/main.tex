\documentclass[11pt, letterpaper]{article}

\usepackage{lineno}
\modulolinenumbers[5]
\usepackage{amsmath}
\usepackage{amssymb}
\usepackage{amsfonts}
\usepackage{fontawesome}
\usepackage{float}
\usepackage{xcolor}
\newcommand{\tens}[1]{\mathsf{#1}}
\usepackage{authblk}

\usepackage{palatino} % Use the Palatino font

\newcommand{\notesRemy}[1]{\textcolor{purple}{\bf \\  Remy: #1}}
\newcommand{\notesEdvard}[1]{\textcolor{blue}{\bf \\ Edvard: #1}}
\newcommand{\notesAriel}[1]{\textcolor{green}{\bf \\ Ariel: #1}}
\newcommand{\notesSimon}[1]{\textcolor{green}{\bf \\ Simon: #1}}

%% Language and font encodings
\usepackage[english]{babel}
\usepackage[utf8x]{inputenc}
\usepackage[T1]{fontenc}

%% Sets page size and margins
\usepackage[a4paper,top=3cm,bottom=2cm,left=3cm,right=3cm,marginparwidth=1.75cm]{geometry}

%% Useful packages
\usepackage{graphicx}
\usepackage[colorinlistoftodos]{todonotes}
\usepackage[colorlinks=true, allcolors=blue]{hyperref}

%% Group authors per affiliation:
\author[1]{Ariel Goobar}
\author[1]{Edvard M{\"o}rsell}
\author[1]{R\'emy Joseph}
\author[2]{Simon Birrer}
\affil[1]{The Oskar Klein Centre, Department of Physics, Stockholm University, AlbaNova, SE-10691 Stockholm, Sweden}
\affil[2]{Kavli Institute for Particle Astrophysics and Cosmology and Department of Physics, Stanford University, Stanford, CA 94305, USA}


% Guidelines for writing this proposal are here: https://www.nordita.org/organizers/proposal/guidelines/index.php 
\begin{document}

\title{Strong Gravitational lensing in the age of time domain, large scale surveys}
\maketitle

\begin{abstract}
    Strong gravitational lensing is a powerful multi-faceted tool for cosmology and astrophysics. Euclid, Rubin and Roman are due to provide there first data within this decade. The unprecedented depth, area and resolution these surveys will bring are expected to raise the sample of gravitational lens systems to a new regime in term of number of samples and data quality. It is paramount that our analysis pipelines  be ready to process these new datasets.
    This program is aimed at bringing together experts of strong gravitational lensing as well as newcomers to the field to brainstorm, learn and work on the new challenges posed by large volumes of time domain strong lensing data.
    
    
\end{abstract}


\section{Aims and Motivation}
    Strong gravitational lensing has become a powerful tool for cosmology as a competitive method for constraining the expansion rate of the Universe, the Hubble constant $H_0$, but also as a probe for dark matter distribution at small scales. Today we only know $\sim 1000$ strong gravitational lens objects. With the advent of deep, multi-band, high resolution surveys ran over long timescales, we expect the number of strong lenses to increase by at least an order of magnitude. The Dark Energy Survey (DES) already yielded hundreds of new lens candidates and only covered 5000 $deg^2$ down to magnitude $\sim 24$. The Legacy Survey of Space and Time (LSST) at the Vera C Rubin Observatory will scan the southern sky (26000 $deg^2$) in six band-pass filters every few days for 10 years. The resulting full depth images will reach down to magnitude $27$. In the same time frame, space experiments like the Roman and Euclid telescopes are expected to image $\sim 15000$ $deg^2$ of the sky with high resolution imaging and spectroscopy in the optical and infrared with partial overlaps with the Rubin footprint. This unprecedented volume and data quality will revolutionize the field of cosmology and in particular through the lens of strong gravitational lenses. 
    
    Indeed, strong gravitational lensing is a powerful multifaceted that requires expertise at various levels to leverage all the information contained in each system.
    Strongly lensed variable sources with measurable time delays are a proven mean of computing $H_0$ that does not rely on anchoring in either local or CMB measurements. So far only a few systems of lensed quasars are known for which light curves are available and two instances of lensed supernovae have been observed. The next generation of surveys is expected to raise these numbers by at least an order of magnitude and by doing so, tighten the constraints on $H_0$. 
    Each strongly lensed system brings a new set of lens and source galaxies. Each lens is a probe of the distribution of dark matter at the galaxy or cluster scale. Lensed sources magnified by lensing open windows on the distant Universe and allows to study galaxy that would otherwise be too faint to exploited. 
    The increased number of lenses also increases the chances for observing rare lensing events like caustic crossings and lensed gravitational waves that could respectively contribute to constraining dark matter distribution and observe gravitational waves visible counter parts.
    
    This wide range of science applications for strong gravitational lenses and the increased discovery rate allowed by multiple instruments calls for community-wide efforts to develop scalable and reliable pipelines that will take us from the data to cosmology results. Furthermore, the complementarities between instruments and surveys calls for coordinated efforts to develop joint analyses of surveys. This presents challenges at many different levels from data processing to data sharing policies to statistical models building. These can be best addressed as a group by gathering a wide and diverse range of strong lensing-motivated scientists.
    
    The aim for this program is to gather the community around this topic and build up expertise across various instruments, methods and probes that all relate to strong gravitational lens analyses. 
    
\section{Budget}
The growing strong lensing community is spread around the globe and a workshop at Nordita would bring people together who otherwise do not have good chances to meet. Hence, we expect to up to half of the participants would come from countries outside of Europe, for which we budget airline tickets for 12 kSEK/person. For participants from Europe, we assume transportation costs of 5 kSEK/person. For 30 participants, we thus anticipate travel costs of $ 15 \times (5 + 12) = 255$ kSEK.
For accommodation we estimate a rate of 1 kSEK/person/night, and we expect to cover hotel costs for a week for 30 participants, i,.e., $7 \times 1 \times 30 = 210$ kSEK for accommodation. We also plan to cover lunch costs ($30 \times 30 \times 0.1 = 90$ kSEK, as well as a conference dinners, 45 kSEK.
The total budget then becomes $210 + 255 + 90 + 45 = 600$ kSEK. In order to meet this maximum budget we anticipate to seek financial help (50 kSEK each) from the Oskar Klein Centre and the Nobel Foundation.
%\notesRemy{I have no idea where to start here}

\section{Nordic Relevance}
The field of strong gravitational lensing in time-domain astronomy was started by 
the pioneering work of Sjur Refsdal in Oslo whom in 1964 showed how the Hubble constant could be measured from time-delays of supernova images. It took more than 50 years before resolved images of lensed SN could be found. The discovery of the first system involving a "standard candle" Type Ia supernova was led by the group in Stockholm, including two of the applicants (Goobar et al 2017; Dhawan et el al 2020, Johansson et al 2020, M\"ortsell et al 2021). Haakon Dahle in Oslo and Jens Hjorth in Copenhagen have led very successful programs studying lensing systems with both cluster and galaxy scale lenses. As the era of large surveys approaches, it would be essential for the Nordic groups to exploit our synergies to make the best use of the precious resources at hand. For that purpose, Goobar and Hjorth are leading an effort along with Spanish colleagues to build an integral field unit MAAT spectrograph to be mounted at the 10-meter telescope in the Canary Islands to perform exquisite measurements of lensing systems, including strongly lensed quasars. The proposed program will be essential to define the science program for that instrument.         
%\notesRemy{I'll need some help from Northerners there}

\section{Tentative program schedule}

    \subsection{Week ?: Winter/Summer school-conference-workshop}
        \notesRemy{We could uses one of the week to have focused conference, workshop or seasonal school. I am more inclined towards a school as the rest of the program already contains talks/workshop content. Plus if we hold it the first day it would help younger participants to engage during the rest of the program.}
    
    \subsection{Week ?: Cosmology with strong gravitational lensing.}
        How do we go from the data to cosmological measurement. How do we understand biases and uncertainties.
    \subsection{Week ?: Time Domain strong gravitational lensing.}
        What is new with large amounts of time domain lensing data? What are the technical challenges. What are the statistical progresses we can expect.
    \subsection{Week ?: Machine Learning and techniques for strong gravitational lensing.}
        What are the technical challenges and solutions we can leverage to address the previous issues?

\section{Provisional list of participants}

\notesRemy{Simon is more in touch with the lensing community than I am, you will surely have good insights their}

\notesSimon{Here a first very incomplete list with short commented-out descriptions of expertise. It's definitely important that we iterate among ourselves and be mindful of what composition makes a productive meeting}

Phil Marshall (SLAC, USA), \\% deputy lead of observations LSST
Aprajita Verma (Oxford, UK), \\% LSST and Euclid
Yashar Hezaveh (Montreal, CAN), \\% DESC SL topical group co-convenor
Thomas Collett (Portsmouth, UK), expressed interest\\% LSST and Euclid
Graham Smith (Birmingham, UK), \\% LSST-UK and SLSC co-chair
Timo Anguita (Santiago, Chile), \\% LSST, SLSC co-chair
Sherry Suyu (Munich, GER), \\% LSST and Euclid
Frederic Courbin (EPFL, CH), \\% Euclid and LSST
Anowar Shajib (Chicago, USA), \\% LSST
Alessandro Sonnenfeld (Leiden, NED), \\% Euclid
Anupreeta More (IUCAA, IND), \\% LSST in-kind
Masamune Oguri (Tokyo, JPN), \\
Jean-Paul Kneib (EPFL, CH),\\
Johan Richard (Lyon, FR), \\
Jens Hjorth (Copenhagen, DK), \\
Haakon Dahle (Oslo, N), \\
Priyamvada Natarajan (Yale, US) \\
Brian Nord (Fermi Lab, US) \\% Rubin
Laurence Perrreault-Levasseur (U Montreal, CA) \\% ML models
Karina Rojas (Portsmouth, UK) \\% DES, CPHIS

\bibliography{references}
\bibliographystyle{IEEEtran}


\end{document}
