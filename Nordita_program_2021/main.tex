\documentclass[11pt, letterpaper]{article}

\usepackage{lineno}
\modulolinenumbers[5]
\usepackage{amsmath}
\usepackage{amssymb}
\usepackage{amsfonts}
\usepackage{fontawesome}
\usepackage{float}
\usepackage{xcolor}
\newcommand{\tens}[1]{\mathsf{#1}}
\usepackage{authblk}

\usepackage{palatino} % Use the Palatino font

\newcommand{\notesRemy}[1]{\textcolor{purple}{\bf \\  Remy: #1}}
\newcommand{\notesEdvard}[1]{\textcolor{blue}{\bf \\ Edvard: #1}}
\newcommand{\notesAriel}[1]{\textcolor{green}{\bf \\ Ariel: #1}}
\newcommand{\notesSimon}[1]{\textcolor{green}{\bf \\ Simon: #1}}

%% Language and font encodings
\usepackage[english]{babel}
\usepackage[utf8x]{inputenc}
\usepackage[T1]{fontenc}

%% Sets page size and margins
\usepackage[a4paper,top=3cm,bottom=2cm,left=3cm,right=3cm,marginparwidth=1.75cm]{geometry}

%% Useful packages
\usepackage{graphicx}
\usepackage[colorinlistoftodos]{todonotes}
\usepackage[colorlinks=true, allcolors=blue]{hyperref}

%% Group authors per affiliation:
\author[1]{R\'emy Joseph}
\author[2]{Simon Birrer}
\author[1]{Ariel Goobar}
\author[1]{Edvard M{\"o}rsell}
\affil[1]{The Oskar Klein Centre, Department of Physics, Stockholm University, AlbaNova, SE-10691 Stockholm, Sweden}
\affil[2]{Simon's affil}

\begin{document}

\title{Strong Gravitational lensing in the age of time domain, large scale surveys}
\maketitle

\begin{abstract}
    Strong gravitational lensing is a powerful multi-faceted tool for cosmology and astrophysics. Recent and upcoming surveys are bring the sample of gravitational lens systems to a new regime in term of number of samples and data quality. 
    This program is aimed at bringing together experts of strong gravitational lensing as well as newcomers to the field to brainstorm, learn and work on the new challenges posed by large volumes of time domain strong lensing data.
    
    
\end{abstract}


\section{Aims and Motivation}
    Strong gravitational lensing has become a powerful tool for cosmology as a competitive method for constraining the expansion rate of the Universe $H_0$, but also as a probe for dark matter distribution at small scales. Today we only know $\sim 1000$ strong gravitational lens objects. With the advent of large scale space based surveys like Euclid and Roman, or the deep ground-based observations over long time frames like Rubin, we expect the number of strong lensing to increase by at least an order of magnitude. In particular, we expect to find many more rare lensing events like multiple-plane strong lenses, multiply imaged supernovae or caustic-crossing events that have the potential to 
    
\section{Budget}

\notesRemy{I have no idea where to start here}

\section{Nordic Relevance}

\notesRemy{I'll need some help from Northerners there}

\section{Tentative program schedule}

    \subsection{Week ?: Winter/Summer school-conference-workshop}
        \notesRemy{We could uses one of the week to have focused conference, workshop or seasonal school. I am more inclined towards a school as the rest of the program already contains talks/workshop content. Plus if we hold it the first day it would help younger participants to engage during the rest of the program.}
    
    \subsection{Week ?: Cosmology with strong gravitational lensing.}
        How do we go from the data to cosmological measurement. How do we understand biases and uncertainties.
    \subsection{Week ?: Time Domain strong gravitational lensing.}
        What is new with large amounts of time domain lensing data? What are the technical challenges. What are the statistical progresses we can expect.
    \subsection{Week ?: Machine Learning and techniques for strong gravitational lensing.}
        What are the technical challenges and solutions we can leverage to address the previous issues?

\section{Provisional list of participants}

\notesRemy{Simon is more in touch with the lensing community than I am, you will surely have good insights their}

\bibliography{references}
\bibliographystyle{IEEEtran}


\end{document}
