\documentclass[11pt, letterpaper]{article}

\usepackage{lineno}
\modulolinenumbers[5]
\usepackage{amsmath}
\usepackage{amssymb}
\usepackage{amsfonts}
\usepackage{fontawesome}
\usepackage{float}
\usepackage{xcolor}
\newcommand{\tens}[1]{\mathsf{#1}}
\usepackage{authblk}

\usepackage{palatino} % Use the Palatino font

\newcommand{\notesRemy}[1]{\textcolor{purple}{\bf \\  Remy: #1}}
\newcommand{\notesEdvard}[1]{\textcolor{blue}{\bf \\ Edvard: #1}}
\newcommand{\notesAriel}[1]{\textcolor{green}{\bf \\ Ariel: #1}}
\newcommand{\notesSimon}[1]{\textcolor{green}{\bf \\ Simon: #1}}

%% Language and font encodings
\usepackage[english]{babel}
\usepackage[utf8x]{inputenc}
\usepackage[T1]{fontenc}

%% Sets page size and margins
\usepackage[a4paper,top=3cm,bottom=2cm,left=3cm,right=3cm,marginparwidth=1.75cm]{geometry}

%% Useful packages
\usepackage{graphicx}
\usepackage[colorinlistoftodos]{todonotes}
\usepackage[colorlinks=true, allcolors=blue]{hyperref}

%% Group authors per affiliation:
\author[1]{Ariel Goobar}
\author[1]{Edvard M{\"o}rsell}
\author[1]{R\'emy Joseph}
\author[2]{Simon Birrer}
\affil[1]{The Oskar Klein Centre, Department of Physics, Stockholm University, AlbaNova, SE-10691 Stockholm, Sweden}
\affil[2]{Kavli Institute for Particle Astrophysics and Cosmology and Department of Physics, Stanford University, Stanford, CA 94305, USA}


% Guidelines for writing this proposal are here: https://www.nordita.org/organizers/proposal/guidelines/index.php 
\begin{document}

\title{Strong Gravitational lensing in the age of time domain, large scale surveys}
\maketitle

\begin{abstract}
    Strong gravitational lensing is a powerful multi-faceted tool for cosmology and astrophysics. Euclid, Rubin and Roman are due to provide their first data within this decade. The unprecedented depth, area and resolution these surveys will bring are expected to raise the sample of gravitational lens systems to a new regime in term of number of samples and data quality. It is paramount that our analysis pipelines  be ready to process these new datasets.
    This program is aimed at bringing together experts of strong gravitational lensing as well as newcomers to the field to brainstorm, learn and work on the new challenges posed by large volumes of time domain strong lensing data.
    
    
\end{abstract}


\section{Aims and Motivation}
    Strong gravitational lensing has become a powerful tool for cosmology as a competitive method for constraining the expansion rate of the Universe, the Hubble constant $H_0$, but also as a probe for dark matter distribution at small scales. Today we only know $\sim 1000$ strong gravitational lens objects. With the advent of deep, multi-band, high resolution surveys ran over long timescales, we expect the number of good quality strong lens observations to increase by at least an order of magnitude. The Dark Energy Survey (DES) already yielded hundreds of new lens candidates and only covers 5'000 $deg^2$ down to magnitude $\sim 24$. The Legacy Survey of Space and Time (LSST) at the Vera C Rubin Observatory will scan the southern sky (26'000 $deg^2$) in six band-pass filters every few days for 10 years. The resulting full depth images will reach down to magnitude $27$. In the same time frame, space experiments like the Roman and Euclid telescopes are expected to image $\sim 15'000$ $deg^2$ of the sky with high resolution imaging and spectroscopy in the optical and infrared with partial overlaps with the Rubin footprint. This unprecedented volume and data quality will revolutionize the field of cosmology and in particular through the lens of strong gravitational lenses. 
    
    Strong gravitational lensing is a powerful multifaceted tool that requires expertise at various levels to leverage all the information contained in each system.
    Strongly lensed variable sources with measurable time delays are a proven mean to infer $H_0$ that does not rely on anchors in either local or CMB measurements. So far only a few systems of lensed quasars are known for which light curves over a $~10$ year time span are available and three instances of lensed supernovae have been observed. The next generation of surveys is expected to raise these numbers by at least an order of magnitude and by doing so, tighten the constraints on $H_0$. 
    
    In gravitational lenses, the measured convergence profiles of lens galaxies are probes of dark matter distribution at the galaxy or cluster scale. 
    Source galaxies, magnified by lensing open windows on the distant Universe and allow us to study high-redshift galaxies that would otherwise be too faint to be observed. 
    The increasing sample size and data quality of lens systems in the upcoming decade will increase the statistical power of these measurements and with it, our understanding of the Universe.
    
    The increased number of lenses also increases the chances for observing rare lensed transient events like caustic crossings and lensed gravitational waves that could respectively contribute to constraining dark matter distribution and observe visible counter parts of gravitational waves further away.
    
    This wide range of science applications for strong gravitational lenses and the increased discovery rate allowed by multiple instruments calls for community-wide efforts to develop scalable and reliable pipelines that will take us from the data to cosmology results. Furthermore, the complementarities between instruments and surveys calls for coordinated efforts to develop joint analyses of surveys. This presents challenges at many different levels, from data processing to data sharing policies to statistical model building. These challenges can be best addressed as a group by gathering a wide and diverse range of strong lensing-motivated experts. Efforts in the direction of joint survey processing planning have already started. for instance  Rubin and Euclid, have formed a \href{https://www.lsst.org/news/rubin-meets-euclid-towards-joint-derived-data-products-working-group}{Joint Derived Data Products Working Group} that has identified joint data products that would take advantage of both surveys' characteristics. Several of these data products are meant to address strong gravitational lensing topics.
    
    The aim for this program is to gather the community around this topic and build up methods, probes and expertise across various instruments,  that all relate to strong gravitational lens analyses. 
    
\section{Budget}
The growing strong lensing community is spread around the globe and a workshop at Nordita would bring people together who otherwise do not have good chances to meet. Hence, we expect and aim for about half of the participants would come from countries outside of Europe, for which we budget airline tickets for 12 kSEK/person. For participants from Europe, we assume transportation costs of 5 kSEK/person. For 30 participants, we thus anticipate travel costs of $ 15 \times (5 + 12) = 255$ kSEK.
For accommodation we estimate a rate of 1 kSEK/person/night, and we expect to cover hotel costs for a week for 30 participants, i,.e., $7 \times 1 \times 30 = 210$ kSEK for accommodation. We also plan to cover lunch costs ($30 \times 30 \times 0.1 = 90$ kSEK, as well as a conference dinner, 45 kSEK.
The total budget then becomes $210 + 255 + 90 + 45 = 600$ kSEK. In order to meet this maximum budget we anticipate to seek financial help (50 kSEK each) from the Oskar Klein Centre and the Nobel Foundation.
%\notesRemy{Checks out for me, given the tight budget, I am wondering if a school would incur extra costs?}

\section{Nordic Relevance}
The field of strong gravitational lensing in time-domain astronomy was started by 
the pioneering work of Sjur Refsdal in Oslo who in 1964 proposed that the Hubble constant could be measured from time delays of supernova (SN) images. It took more than 50 years until resolved images of lensed SN were discovered. The discovery of the first system involving a "standard candle" Type Ia supernova was led by the group in Stockholm, including two of the applicants (Goobar et al 2017; Dhawan et el al 2020, Johansson et al 2020, M\"ortsell et al 2021). Haakon Dahle in Oslo and Jens Hjorth in Copenhagen have led very successful programs studying lensing systems with both cluster and galaxy scale lenses. As the era of large surveys approaches, it would be essential for the Nordic groups to exploit our synergies to make the best use of the precious resources at hand. For that purpose, Goobar and Hjorth are leading an effort along with Spanish colleagues to build an integral field unit MAAT spectrograph to be mounted at the 10-meter telescope in the Canary Islands to perform exquisite measurements of lensing systems, including strongly lensed quasars. The proposed program will be essential to define the science program for that instrument.         
%\notesRemy{I'll need some help from Northerners there}

\section{Tentative program schedule, April 2023}

    \subsection{Week 1: Winter/Summer school-conference-workshop}
        This school is meant to provide participants with the tools to engage during the program. We aim at providing background theoretical and technical aspects of cosmology with strong gravitational lensing. The school should also be an opportunity to gather information about the multiple surveys relevant to the field. We hope that by the end of this week, all participants should feel equipped to engage with the larger audience of this program during the rest of their stay. 
    
    \subsection{Week 2: Cosmology with strong gravitational lensing.}
        A week of focus on the theoretical aspects of cosmology to which gravitational lensing is relevant. What areas has gravitational lenses contributed to, and what further advances can we expect to make with this tool.

    \subsection{Week 3: Technical challenges in the age of machine learning.}
        This week will focus on the technical and statistical aspects of strong gravitational lenses analyses, in particular as the scale of our datasets is rapidly growing. Not only will we have more lenses, but we will also have multi-band, multi-instrument and multiple-epoch data at our disposal. This week will be a very timely opportunity to share investigate new technical solutions on how to take advantage of this new paradigm. Machine learning in particular is a very promising tool in this context and it is paramount that we identify how to make the best use of this tool. 
        
    \subsection{Week 4: Joining forces across surveys.}
        New experiments running at large scale on similar time-frames increases opportunities for collaboration and for new measurements based on multiple instruments across time. Gathering expertise is key to taking advantage of this unique combination.

\section{Provisional list of participants}

The following list displays potential participants that would gather broad expertise and foster collaborative work during this program. We reached out to the member of this list to enquire about their interest. For those who replied, the answers were very positive and enthusiastic, and acknowledged the timeliness of the program.

List of potential participants: \\
Phil Marshall (SLAC, USA), replied: expressed interest conditional to family support and childcare, \\% deputy lead of observations LSST
Aprajita Verma (Oxford, UK), \\% LSST and Euclid
Yashar Hezaveh (Montreal, CAN), \\% DESC SL topical group co-convenor
Thomas Collett (Portsmouth, UK), replied: expressed interest,\\% LSST and Euclid
Graham Smith (Birmingham, UK), replied: expressed interest \\% LSST-UK and SLSC co-chair
Timo Anguita (Santiago, Chile), \\% LSST, SLSC co-chair
Sherry Suyu (Munich, GER), replied: expressed interest,\\% LSST and Euclid
Frederic Courbin (EPFL, CH), replied: expressed interest, \\% Euclid and LSST
Anowar Shajib (Chicago, USA), \\% LSST
Alessandro Sonnenfeld (Leiden, NED), \\% Euclid
Anupreeta More (IUCAA, IND), replied: expressed interest, \\% LSST in-kind
Masamune Oguri (Tokyo, JPN), replied: expressed interest,\\
Jean-Paul Kneib (EPFL, CH),\\
Johan Richard (Lyon, FR), \\
Jens Hjorth (Copenhagen, DK), \\
Haakon Dahle (Oslo, N), replied: expressed interest,\\
Priyamvada Natarajan (Yale, US) \\
Brian Nord (Fermi Lab, US), replied: expressed interest, \\% Rubin
Laurence Perrreault-Levasseur (U Montreal, CA) \\% ML models
Karina Rojas (Portsmouth, UK) \\% DES, CPHIS

\bibliography{references}
\bibliographystyle{IEEEtran}


\end{document}
